% Options for packages loaded elsewhere
\PassOptionsToPackage{unicode}{hyperref}
\PassOptionsToPackage{hyphens}{url}
%
\documentclass[
]{article}
\usepackage{amsmath,amssymb}
\usepackage{lmodern}
\usepackage{ifxetex,ifluatex}
\ifnum 0\ifxetex 1\fi\ifluatex 1\fi=0 % if pdftex
  \usepackage[T1]{fontenc}
  \usepackage[utf8]{inputenc}
  \usepackage{textcomp} % provide euro and other symbols
\else % if luatex or xetex
  \usepackage{unicode-math}
  \defaultfontfeatures{Scale=MatchLowercase}
  \defaultfontfeatures[\rmfamily]{Ligatures=TeX,Scale=1}
\fi
% Use upquote if available, for straight quotes in verbatim environments
\IfFileExists{upquote.sty}{\usepackage{upquote}}{}
\IfFileExists{microtype.sty}{% use microtype if available
  \usepackage[]{microtype}
  \UseMicrotypeSet[protrusion]{basicmath} % disable protrusion for tt fonts
}{}
\makeatletter
\@ifundefined{KOMAClassName}{% if non-KOMA class
  \IfFileExists{parskip.sty}{%
    \usepackage{parskip}
  }{% else
    \setlength{\parindent}{0pt}
    \setlength{\parskip}{6pt plus 2pt minus 1pt}}
}{% if KOMA class
  \KOMAoptions{parskip=half}}
\makeatother
\usepackage{xcolor}
\IfFileExists{xurl.sty}{\usepackage{xurl}}{} % add URL line breaks if available
\IfFileExists{bookmark.sty}{\usepackage{bookmark}}{\usepackage{hyperref}}
\hypersetup{
  pdftitle={Modul 3},
  pdfauthor={Eva},
  hidelinks,
  pdfcreator={LaTeX via pandoc}}
\urlstyle{same} % disable monospaced font for URLs
\usepackage[margin=1in]{geometry}
\usepackage{graphicx}
\makeatletter
\def\maxwidth{\ifdim\Gin@nat@width>\linewidth\linewidth\else\Gin@nat@width\fi}
\def\maxheight{\ifdim\Gin@nat@height>\textheight\textheight\else\Gin@nat@height\fi}
\makeatother
% Scale images if necessary, so that they will not overflow the page
% margins by default, and it is still possible to overwrite the defaults
% using explicit options in \includegraphics[width, height, ...]{}
\setkeys{Gin}{width=\maxwidth,height=\maxheight,keepaspectratio}
% Set default figure placement to htbp
\makeatletter
\def\fps@figure{htbp}
\makeatother
\setlength{\emergencystretch}{3em} % prevent overfull lines
\providecommand{\tightlist}{%
  \setlength{\itemsep}{0pt}\setlength{\parskip}{0pt}}
\setcounter{secnumdepth}{-\maxdimen} % remove section numbering
\ifluatex
  \usepackage{selnolig}  % disable illegal ligatures
\fi

\title{Modul 3}
\author{Eva}
\date{9/17/2021}

\begin{document}
\maketitle

Dasar Teori

Variasi tipe data pada R memfasilitasi keberagaman jenis variabel data.
Sebagai contoh, terdapat data yang terdiri dari sekumpulan angka dan
data lain yang berisi sekumpulan karakter. Pada contoh lain, ada pula
data yang berbentuk tabel maupun kumpulan (list) angka sederhana. Dengan
bantuan fungsi class, kita akan mendapatkan kemudahan dalam
mendefinisikan tipe data yang kita miliki:

a \textless- 2 class(a) \#\textgreater{} {[}1{]} ``numeric''

Agar dapat bekerja secara efisien dalam menggunakan bahasa pemrograman
R, penting untuk mempelajari terlebih dahulu tipe data dari
variabel-variabel yang kita miliki sehingga akan mempermudah dalam
penentuan proses analisis data yang dapat dilakukan terhadap
variabelvariabel tersebut.

Data Frames Cara paling umum yang dapat digunakan untuk menyimpan
dataset dalam R adalah dalam tipe data frame. Secara konseptual, kita
dapat menganggap data frame sebagai tabel yang terdiri dari baris yang
memiliki nilai pengamatan dan berbagai variabel yang didefinisikan dalam
bentuk kolom. Tipe data ini sangat umum digunakan untuk dataset, karena
data frame dapat menggabungkan berbagai jenis tipe data dalam satu
objek. Untuk memahami tipe data frame, silahkan mengakses contoh dataset
pada library(dslabs) dan pilih dataset ``murders'' menggunakan fungsi
data:

library(dslabs) data(murders)

Untuk memastikan bahwa dataset tersebut tipenya adalah data frame, dapat
digunakan perintah berikut:

class(murders) \#\textgreater{} {[}1{]} ``data.frame''

Untuk memeriksa lebih lanjut isi dataset, dapat pula digunakan fungsi
str untuk mencari tahu lebih rinci mengenai struktur suatu objek:

str(murders) \#\textgreater{} `data.frame': 51 obs. of 5 variables:
\#\textgreater{} \$ state : chr ``Alabama'' ``Alaska'' ``Arizona''
``Arkansas'' \ldots{} \#\textgreater{} \$ abb : chr ``AL'' ``AK'' ``AZ''
``AR'' \ldots{} \#\textgreater{} \$ region : Factor w/ 4 levels
``Northeast'',``South'',..: 2 4 4 2 4 4 1 2 2 \#\textgreater{} 2
\ldots{} \#\textgreater{} \$ population: num 4779736 710231 6392017
2915918 37253956 \ldots{} \#\textgreater{} \$ total : num 135 19 232 93
1257 \ldots{}

Dengan menggunakan fungsi str, dapat diketahui bahwa dataset ``murders''
terdiri dari 51 baris dan lima variabel: state, abb, region, population,
dan total. Selanjutnya, untuk melihat contoh enam baris pertama pada
dataset, dapat digunakan fungsi head:

head(murders) \#\textgreater{} state abb region population total
\#\textgreater{} 1 Alabama AL South 4779736 135 \#\textgreater{} 2
Alaska AK West 710231 19 \#\textgreater{} 3 Arizona AZ West 6392017 232
\#\textgreater{} 4 Arkansas AR South 2915918 93 \#\textgreater{} 5
California CA West 37253956 1257 \#\textgreater{} 6 Colorado CO West
5029196 65

Untuk analisis awal tiap variabel yang diwakili dalam bentuk kolom pada
tipe data frame, dapat digunakan operator aksesor (\$) dengan cara
berikut:

dari lima variabel yang dapat dievaluasi menggunakan operator aksesor,
sebelumnya, melalui fungsi str, telah kita ketahui bahwa variabel yang
dimiliki dataset adalah: state, abb, region, population, dan total.
Sebagai alternatif, terdapat pula fungsi name, yang dapat digunakan
seperti contoh dibawah ini:

Vector : numeric, character, dan logical Objek
murders\(population terdiri dari sekumpulan numeric atau data-data angka. Sehingga, kita dapat mendefinisikan bahwa tipe data murders\)population
berupa vector. Angka tunggal secara teknis dapat didefinisikan sebagai
vektor dengan panjang 1, tetapi secara umum kita akan menggunakan vector
sebagai istilah untuk merujuk ke objek yang terdiri dari beberapa entri.
Untuk mengidentifikasi banyaknya entri dalam suatu vector dapat
digunakan fungsi length seperti contoh berikut:

head(murders) \#\textgreater{} state abb region population total
\#\textgreater{} 1 Alabama AL South 4779736 135 \#\textgreater{} 2
Alaska AK West 710231 19 \#\textgreater{} 3 Arizona AZ West 6392017 232
\#\textgreater{} 4 Arkansas AR South 2915918 93 \#\textgreater{} 5
California CA West 37253956 1257 \#\textgreater{} 6 Colorado CO West
5029196 65 names(murders) \#\textgreater{} {[}1{]} ``state'' ``abb''
``region'' ``population'' ``total''

Vector : numeric, character, dan logical Objek
murders\(population terdiri dari sekumpulan numeric atau data-data angka. Sehingga, kita dapat mendefinisikan bahwa tipe data murders\)population
berupa vector. Angka tunggal secara teknis dapat didefinisikan sebagai
vektor dengan panjang 1, tetapi secara umum kita akan menggunakan vector
sebagai istilah untuk merujuk ke objek yang terdiri dari beberapa entri.
Untuk mengidentifikasi banyaknya entri dalam suatu vector dapat
digunakan fungsi length seperti contoh berikut:

length(murders\$population) \#\textgreater{} {[}1{]} 51

Vector khusus ini bertipe numeric karena populasi terdiri dari data-data
angka:

class(murders\$population) \#\textgreater{} {[}1{]} ``numeric''

Secara matematis, nilai-nilai dalam murders\$population adalah berupa
integer. Namun, secara default, data angka akan diberikan tipe numeric
meskipun sebenarnya data tersebut merupakan bilangan bulat. Misalnya,
class(1) akan mengidentifikasi nilai 1 sebagai tipe numeric. Untuk
mengubah tipe numeric menjadi integer, dapat digunakan fungsi
as.integer() atau dengan menambahkan L pada akhir data angka, contoh:
1L. Untuk melihat perbedaannya, silahkan gunakan class(1L). Vector juga
dapat digunakan untuk menyimpan string dengan tipe character, Sebagai
contoh:

nama negara pada dataset ``murders'': class(murders\$state)
\#\textgreater{} {[}1{]} ``character''

Jenis vector penting lainnya adalah logical yang nilainya berupa TRUE
atau FALSE

z \textless- 3 == 2 z \#\textgreater{} {[}1{]} FALSE class(z)
\#\textgreater{} {[}1{]} ``logical''

Factors Dalam dataset ``murders'', variabel state yang berisi data
karakter bukan bertipe vector: character, namun, tipe datanya adalah
factor:

class(murders\$region) \#\textgreater{} {[}1{]} ``factor''

Faktor berguna untuk menyimpan data kategorikal. Dapat dilihat, bahwa
hanya terdapat 4 wilayah pada variabel state. Untuk melihat jumlah
kategori yang dimiliki oleh variabel dengan tipe data factor dapat
digunakan fungsi level:

levels(murders\$region) \#\textgreater{} {[}1{]} ``Northeast'' ``South''
``North Central'' ``West''

Pada background process, R menyimpan level sebagai bilangan bulat yang
memiliki peta tersendiri untuk melacak arti label dari bilangan
tersebut. Hal ini dimaksudkan untuk penghematan memori, terutama apabila
karakter dari tiap level cukup panjang. Standarnya, level akan
ditampilkan sesuai urutan abjad.

Lists Data frame merupakan sekumpulan list yang memiliki kelas yang
berbeda-beda. Sama halnya dengan data frame, analisis list dapat
dilakukan dengan menggunakan operator aksesor (\$) dan dua kurung siku
({[}{[}).

Matriks Matriks merupakan tipe data yang mirip dengan data frame karena
keduanya memiliki dua dimensi, yaitu: baris dan kolom. Namun, sama
halnya dengan tipe data vector numerik, karakter dan logis, entri dalam
matriks harus terdiri dari jenis vector yang sama. Dalam hal ini, data
frame dapat dikatakan sebagai tipe data yang paling cocok untuk
menyimpan data, karena kita dapat memiliki karakter, faktor, dan angka
sekaligus dalam satu data frame. Namun matriks memiliki satu keunggulan
yang tidak dimiliki oleh tipe data frame: pada matriks dapat dilakukan
operasi aljabar.

Untuk mendefinisikan matriks, dapat digunakan fungsi matrix dengan
mendefinisikan pula argumen berupa jumlah baris dan kolom yang
diinginkan.

mat \textless- matrix(1:12, 4, 3) mat \#\textgreater{} {[},1{]} {[},2{]}
{[},3{]} \#\textgreater{} {[}1,{]} 1 5 9 \#\textgreater{} {[}2,{]} 2 6
10 \#\textgreater{} {[}3,{]} 3 7 11 \#\textgreater{} {[}4,{]} 4 8 12

Untuk mengakses entri tertentu dalam matriks, dapat digunakan tanda
kurung siku ({[}). Sebagai contoh, kita akan menampilkan data pada baris
kedua, kolom ketiga, menggunakan:

mat{[}2, 3{]} \#\textgreater{} {[}1{]} 10

\end{document}
